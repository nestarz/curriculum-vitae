%% start of file `template.tex'.
%% Copyright 2006-2010 Xavier Danaux (xdanaux@gmail.com).
%% Copyright 2010-2011 Mark Liu (markwayneliu@gmail.com).
%
% This work may be distributed and/or modified under the
% conditions of the LaTeX Project Public License version 1.3c,
% available at http://www.latex-project.org/lppl/.

\documentclass[11pt,a4paper,sans]{moderncv}

\usepackage{verbatim}

% moderncv themes
\moderncvstyle{classic}
\moderncvcolor{blue}

% character encoding
\usepackage[utf8]{inputenc}                   % replace by the encoding you are using

% adjust the page margins
\usepackage[scale=0.8]{geometry}
%\setlength{\hintscolumnwidth}{3cm}						% if you want to change the width of the column with the dates
%\AtBeginDocument{\setlength{\maketitlenamewidth}{6cm}}  % only for the classic theme, if you want to change the width of your name placeholder (to leave more space for your address details
%\AtBeginDocument{\recomputelengths}                     % required when changes are made to page layout lengths


% personal data
\firstname{Elias}
\familyname{RHOUZLANE}
\title{Etudiant en Sciences Cognitives, 20 ans}      
\address{29 Bis Impasse du Taillan}{33320 Eysines, Aquitaine, France}    % optional, remove the line if not wanted
\mobile{06 20 40 62 67}                    % optional, remove the line if not wanted
\email{elias.rhouzlane@gmail.com}                      % optional, remove the line if not wanted
%\homepage{http://www.elias.cognicorn.com}                % optional, remove the line if not wanted


% to show numerical labels in the bibliography; only useful if you make citations in your resume
%\makeatletter
%\renewcommand*{\bibliographyitemlabel}{\@biblabel{\arabic{enumiv}}}
%\makeatother

\nopagenumbers{}                             % uncomment to suppress automatic page numbering for CVs longer than one page
%----------------------------------------------------------------------------------
%            content
%----------------------------------------------------------------------------------
\begin{document}
\maketitle

\section{Formation}
\cventry{2012--2015}{Licence Mathématiques et informatique appliquées aux sciences humaines et sociales, Parcours Sciences Cognitives}{Mention Bien, Université de Bordeaux}{Bordeaux}{}{}
\cventry{2009--2012}{Baccalauréat Scientifique-SVT}{Mention Bien, Lycée Camille Jullian}{Bordeaux}{}{}  % arguments 3 to 6 can be left empty


\section{Expériences}

\cventry[-1em]{2013--2015}{Tuteur pédagogique spécialisé en Informatique}{Université de Bordeaux}{Bordeaux}{}{
}
\cventry{}{Tuteur pédagogique spécialisé en Mathématiques}{Université de Bordeaux}{Bordeaux}{}{
Soutien en mathématiques et informatique à raison de deux heures par semaine durant lesquelles j'apportais une aide aux étudiants de première année sur des notions mal comprises.
}
\cventry{2012--2013}{Stagiaire au Laboratoire « Cognition et Facteurs Humains »}{Université de Bordeaux}{}{}{
Découverte de la recherche en psychologie cognitive et passation de test neuropsychologique.
}

\section{Compétences}
\subsection{Langues}
\cvlanguage{Français}{Langue maternelle}{}
\cvlanguage{Anglais}{Lu, parlé, écrit — Niveau C2}{}
\cvlanguage{Espagnol}{Notions}{}
\subsection{Informatique}
\cvcomputer{Langages}{Python, C\#, Octave, Mapple, SciLab, R, SQL, JSON, PHP, HTML, CSS}{Technologies}{OpenCV, Panda3D, Unity, OpenViBE, MySQL, \LaTeX{}, Bash, Ubuntu, Windows, Bootstrap, Photoshop, Office}
%\cvcomputer{langages}{Javascript, Octave, R, C++, C\#, HOC}{Technologies}{OpenCV, MeshLab, Panda3D, Unity, Ubuntu, Windows, Illustrator, RPGMaker, NEURON}%

\section{Activités}
\cventry{2014--Aujourd'hui}{Bénévole au pôle Web/Design et Jour-J}{TEDxUniversitéDeBordeaux}{Bordeaux}{}{
L'objectif de TED et des événements TEDx est de diffuser et vulgariser des idées habituellement vues comme complexes. J'ai notamment apporté mon aide à la conception du diaporama des intervenants ainsi qu'à l'affiche publicitaire de l'événement.
}
\cventry{2013--Aujourd'hui}{Responsable Graphiste et Web}{Association des étudiants en Sciences Cognitives et Ergonomie de Bordeaux dit Ascoergo}{Bordeaux}{}{
Participe à l'organisation d'événements (CogTalks, WAF, Table ronde des anciens, ...), à la création de la nouvelle identité graphique ainsi qu'à la conception d'un nouveau site internet dédié. Je suis aussi en charge de la communication visuelle.
}
\cventry{Mars 2015}{Bénévole Logistique et Jour-J}{Forum des Sciences Cognitives}{Paris}{}{
}
\cventry{}{Centres d'intêret}{}{}{}{
Programmation, Animation et bande dessinée japonaise, Cosplay, Littérature de fantaisie et science fiction, Jeux vidéo compétitifs, Jeu de rôle, Nouvelles technologies, Pratique de sport (Ju-jitsu, running)}

%\section{Cours Pertinents}
%\subsection{Université de Bordeaux, Aquitaine}
%\cvline{Informatique}{Programmation en Python, Imagerie Numérique, Informatique Théorique, Connaissances et Représentations, Conception de Site Web Dynamiques, Algorithme et Structures de Données}
%\cvline{Mathématiques}{Analyse de Données, Modélisation Statistiques, Systèmes Dynamiques, Processus Markoviens, Méthodes Numériques Linéraires et Non Linéaires, Algèbre, Analyse}
%\cvline{Sciences Cognitives}{Modélisation Neuronale, Neurosciences Computationelles, Perception et Action, Langage et Cognition, Bases en Neurobiologie et Neuroanatomie, Fondamentaux Cognitifs}
\end{document}
